\begin{abstract}
    In questa tesi viene presentata un'esposizione introduttiva sulla geometria informativa della meccanica quantistica. La trattazione inizia dalla geometria informativa classica, definendo le varietà statistiche e la metrica informativa di Fisher e, infine, dimostrando la disuguaglianza di Cramér-Rao. Successivamente, viene introdotta la formulazione geometrica della meccanica quantistica: partendo dai postulati nella loro formulazione vettoriale, vengono definiti gli operatori di densità e viene mostrato come il prodotto interno dello spazio di Hilbert induca la metrica di Fubini-Study sullo spazio proiettivo di Hilbert. Nell'ultima parte, le due descrizioni geometriche vengono collegate definendo l'informazione quantistica di Fisher come generalizzazione di quella classica e trovando che coincide, a meno di un fattore costante, con la metrica di Fubini-Study. Si conclude utilizzando questa descrizione per dimostrare la versione quantistica della disuguaglianza di Cramér-Rao, risultato centrale della teoria della stima quantistica (QET).
\end{abstract}