\section{Quantum estimation}
\subsection{Variance and expectation value}
The analogy between quantum states and statistical models we developed allows us to state and prove the quantum versions of the results in \cref{ch:param-est} about classical parameter estimation.

We start with the quantum version of \cref{th:gradient}.
\begin{theorem}
    For any observable $A\in\mathfrak{A}$, we have that
    \begin{equation}
        (\mathrm{grad}\,\mathrm{E}[A])_\rho^{(e)}=(A-\mathrm{E}_\rho[A])^{(\rho)} \qquad\forall \rho\in\mathbf{P}\mathcal{H}
    \end{equation}
    where the gradient is the dual tangent vector of the differential with respect to the quantum Fisher metric.
\end{theorem}
\begin{proof}
    For every $\Tilde{X}_\theta\in T_\rho\mathbf{P}\mathcal{H}$, we have
    \begin{align*}
        \Tilde{X}_\theta(\mathrm{E}[A])&=\mathrm{tr}(\Tilde{X}_\theta(\rho)\,A)
        =\mathrm{tr}(d\rho_\theta\,A)
        \\
        &=\mathrm{tr}\left(d\rho_\theta\,(A-\mathrm{E}_\rho[A])\right)
        \\
        &=\mathrm{tr}\left(\frac{1}{2}\left\{\rho,\,L^{(\rho)}_\theta\right\}\,(A-\mathrm{E}_\rho[A])\right)
        \\
        &=\frac{1}{2}\mathrm{tr}\left(\rho\,L^{(\rho)}_\theta\,(A-\mathrm{E}_\rho[A])+L^{(\rho)}_\theta\,\rho\,(A-\mathrm{E}_\rho[A])\right)
        \\
        &=\llangle L^{(\rho)}_\theta,(A-\mathrm{E}_\rho[A]) \rrangle_\rho=\llangle L^{(\rho)}_\theta,(A-\mathrm{E}_\rho[A])^{(\rho)} \rrangle_\rho
    \end{align*}
\end{proof}
Then it also follows that
\begin{corollary}
    For any observable $A\in\mathfrak{A}$,
    \begin{equation}
        \mathrm{V}_\rho[A]= \lVert (d\mathrm{E}[A])_\rho \rVert_\rho^2
    \end{equation}
\end{corollary}

We can now consider pure state models, i.e., submanifolds $\mathcal{S}$ of $\mathbf{P}\mathcal{H}$, and, by the same arguments we used in the classical case, we have that
\begin{theorem}
    Given a pure state model $\mathcal{S}$, for any observable $A$, we have that
    \begin{equation}
            \mathrm{V}_\rho[A]\ge \lVert (\left.d\mathrm{E}[A]\right\rvert_\mathcal{S})_\rho \rVert_\rho^2
    \end{equation}
    where the equality holds if and only if
    \begin{equation}
        (A-\mathrm{E}_\rho[A])^{(\rho)}\in T_\rho\mathcal{S}
    \end{equation}
\end{theorem}

\subsection{Quantum Cramér-Rao bound}

We can now develop the quantum version of the Cramér-Rao bound for statistical models. We will proceed similarly to the classical version, but we will have to be careful when treating multivariate models.

Let us consider an m-dimensional pure state model
\begin{equation}
    \mathcal{S}=\{\rho_\xi\mid\xi=[\xi^{(1)},\dots,\xi^{(m)}]\in\Xi\subseteq\mathbb{R}^n\}
\end{equation}
where $\Xi$ is the parameters set. We can define the unbiased estimators for the single parameters as the observables
\begin{equation}
    F^{(i)}\in\mathfrak{A}:\quad \mathrm{E}_\rho[F^{(i)}]=\xi^{(i)}\qquad\forall\rho\in\mathcal{S}
\end{equation}
Then, if we consider an m-tuple of single-parameter estimators
\begin{equation}
    \Vec{F}=[F^{(1)},\dots,F^{(m)}]
\end{equation}
we can define the matrix $W_\xi[\Vec{F}]=\{w^{ij}_\xi\}$ where
\begin{equation}
    w^{ij}_\xi\coloneqq \mathrm{Cov}_{\rho_\xi}[F^{(i)},F^{(j)}]=\llangle F^{(i)}-\xi^{(i)},F^{(j)}-\xi^{(j)} \rrangle_{\rho_\xi}
\end{equation}
so that on the diagonal we have the variances of the estimators of the individual parameters
\begin{equation}
    w^{ii}_\xi=\mathrm{V}_{\rho_\xi}[F^{(i)}]
\end{equation}
We can then repeat the same arguments as for \cref{th:cramer-rao} to prove the following theorem.
\begin{theorem}
    Let $\mathcal{S}=\left\{ \rho_\xi \mid \xi\in\Xi \right\}$ be an m-dimensional pure state model of $\mathbf{P}\mathcal{H}$. Then, for any m-tuple $\Vec{F}$ of single-parameter estimators, the matrix $\mathrm{W}_\xi[\Vec{F}]$ satisfies
    \begin{equation}
        \mathrm{W}_\xi[\Vec{F}]\ge G_F^{-1}(p_\xi)
    \end{equation}
    in the sense that $\mathrm{W}_\xi[\Vec{F}]-G_F^{-1}(p_\xi)$ is positive semi-definite.
\end{theorem}

Differently from the classical case, the m-tuple of single-parameter estimators we defined cannot be used as a multivariate estimator. Intuitively, this is because every single-parameter estimator is a PVM and, unless the corresponding observables commute, the order of the measurements changes the distributions of the outcomes.

A general quantum estimator is composed of a POVM $\hat{M}$ with outcomes in a set $\mathcal{X}$ and of a classical estimator $\hat{\xi}$ such that
\begin{equation}
    \hat{M}=\{(M_x,x)\}_{x\in\mathcal{X}}\quad\mathrm{and}\quad\hat{\xi}:\mathcal{X}\to\Xi
\end{equation}
This definition can be interpreted as performing a quantum measurement on a system and then classically processing the results in order to estimate the parameters of the original quantum state. Then, by denoting
\begin{equation}
    \mathrm{E}_\rho[\hat{\xi}]\coloneqq\sum_{x\in\mathcal{X}}\hat{\xi}(x)p_\rho(x)=\sum_{x\in\mathcal{X}}\hat{\xi}(x)\mathrm{tr}(\rho\,M_x)
\end{equation}
we say that a quantum estimator $(\hat{M},\hat{\xi})$ is unbiased when
\begin{equation}
    \mathrm{E}_{\rho_\xi}[\hat{\xi}]=\xi\qquad\forall \rho_\xi\in\mathcal{S}
\end{equation}
The variance-covariance matrix of the estimator is then $\mathrm{V}_\xi[(\hat{M},\hat{\xi})]=\{v^{ij}_\xi\}$ where
\begin{equation}
    v^{ij}=\sum_{x\in\mathcal{X}}(\hat{\xi}^{(i)}(x)-\xi^{(i)})(\hat{\xi}^{(j)}(x)-\xi^{(j)})\mathrm{tr}(\rho\,M_x)
\end{equation}
and the following lemma holds.
\begin{lemma}
    Let $(\hat{M},\hat{\xi})$ be an unbiased estimator for a pure state model. Then
    \begin{equation}
        F^{(i)}=\sum_{x\in\mathcal{X}}\xi^{(i)}(x)M_x
    \end{equation}
    defines an m-tuple $\Vec{F}$ of unbiased single-parameter estimators. Moreover, the following inequality holds
    \begin{equation}
        \mathrm{V}_\xi[(\hat{M},\hat{\xi})]\ge\mathrm{W}_\xi[\Vec{F}]
    \end{equation}
    where the equality is satisfied if and only if $\hat{M}$ is a PVM.
\end{lemma}
\begin{proof}
    \todo{proof? A complete one is in [Amari]}
\end{proof}

We can thus finally state the quantum version of the Cramér-Rao bound, also known as the \emph{Holevo-Helstrom} theorem.
\begin{theorem} [Quantum Cramér-Rao bound]
    Let $(\hat{M},\hat{\xi})$ be an unbiased estimator for a pure state model $\mathcal{S}=\left\{ \rho_\xi \mid \xi\in\Xi \right\}$ of $\mathbf{P}\mathcal{H}$. Then its variance-covariance matrix satisfies
    \begin{equation}
        \mathrm{V}_\xi[(\hat{M},\hat{\xi})]\ge G_{QF}^{-1}(\rho_\xi)=\frac{1}{4}\,G_{FS}^{-1}(\rho_\xi)
    \end{equation}
    where $G_{QF}$ and $G_{FS}$ are the matrix representations, respectively, of the \emph{quantum Fisher metric} and of the \emph{Fubini-Study metric} of $\mathbf{P}\mathcal{H}$.
\end{theorem}

This important result sets an intrinsic limit to the amount of information retrievable from the state of a quantum system.
