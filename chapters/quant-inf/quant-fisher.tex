\section{Quantum information geometry}
\subsection{Observables and the tangent space}

As we have seen in \cref{sec:postulates}, quantum states can be interpreted as a generalization of probability distributions. We now want to further investigate the parallelisms between the geometry of quantum states and the one of probability distributions. 

We start by recalling the representation of tangent vectors as linear operators, from \cref{eq:tgt-operator} we have
\begin{equation}
    \Tilde{X}_\theta\equiv\frac{\partial}{\partial\theta}\in T_\rho\myPH \mapsto d\rho_\theta\in T_\rho\mathfrak{D}\quad\mathrm{where}\quad d\rho_\theta=\left.\frac{d}{d\theta}\left[\rho(\theta)\right]\right\rvert_{\theta=0}
\end{equation}
which is analogous to the mixture representation of tangent vectors to probability distributions we defined in \cref{eq:mixt-rep}, so that
\begin{gather} 
    \Tilde{X}_\theta^{(m)}\equiv d\rho_\theta\qquad\forall\Tilde{X}_\theta\in T_\rho\myPH
    \\
    \mathrm{and}\quad T_\rho^{(m)}\myPH\equiv T_\rho\mathfrak{D}
\end{gather}
In the last paragraph of \cref{ch:fubini-study} it was shown that any tangent vector $d\rho\in T_\rho\mathfrak{D}$ can be expressed as
\begin{equation} \label{eq:QI-mixt-rep}
    d\rho = i[A,\rho]
\end{equation}
for some Hermitian operator $A$, thus  $d\rho$ is also Hermitian and we may interpret it as an observable. Further, we have
\begin{equation}
    \mathrm{tr}(d\rho_\theta)=\frac{d}{d\theta}\left[\mathrm{tr}(\rho(\theta))\right]=0
\end{equation}
that is analogous to \cref{eq:mixt-zero-sum}.

To define the exponential representation of tangent vectors, we would have to divide the mixture representation by the probability distribution itself. This is not possible since we are dealing with operators, instead, we can use the implicit expression of \cref{eq:rep-rel} and symmetrize it to obtain
\begin{equation} \label{eq:SLD-mixture}
    d\rho_\theta = \frac{1}{2}\left[\rho\,L^{(\rho)}_\theta+L^{(\rho)}_\theta\,\rho\right]
\end{equation}
To precisely define this operator we express any $d\rho\in T_\rho\mathfrak{D}$ as in \cref{eq:QI-mixt-rep}, so that we obtain the following expression
\begin{equation}
    i[A,\rho] = \frac{1}{2}\left\{\rho,L^{(\rho)}(A)\right\}
\end{equation}
It can be proven that when $\rho$ is Hermitian and positive semi-definite $L^{(\rho)}(A)$ is determined for every operator $A$, up to a term that anticommutes with $\rho$. Then $L^{(\rho)}(A)$ is known as the \emph{symmetric logarithmic derivative (SLD)} and one can also show that if $A$ is Hermitian $L^{(\rho)}(A)$ is also Hermitian. Analogously with probability distributions, the expectation value of the SLD is zero
\begin{equation}
    \mathrm{E}\left[L^{(\rho)}\right]=\mathrm{tr}(\rho\,L^{(\rho)})=\frac{1}{2}\mathrm{tr}(\rho\,L^{(\rho)}+L^{(\rho)}\,\rho)=\mathrm{tr}(d\rho)=0
\end{equation}
reguardless of the anticommutating term.

For pure density operators, we can find an explicit expression for the SLD as follows
\begin{equation}
    d\rho_\theta=\left.\frac{d}{d\theta}\left[\rho(\theta)\right]\right\rvert_{\theta=0}=\left.\frac{d}{d\theta}\left[\rho^2(\theta)\right]\right\rvert_{\theta=0}=\rho\,d\rho_\theta+d\rho_\theta\,\rho
\end{equation}
so that we can set
\begin{equation}\label{eq:qm-exp-mix}
    L^{(\rho)}_\theta=2d\rho_\theta\qquad\forall d\rho_\theta\in T_\rho\mathfrak{D}
\end{equation}
by implicitly fixing the anticommutating term. It will become clear in the next sections that this choice has no effect on the geometry we will develop. 

We thus have that $L^{(\rho)}_\theta$ is analogous to the exponential representation of $d\rho_\theta$
\begin{equation}
    \Tilde{X}_\theta^{(e)}\equiv L^{(\rho)}_\theta\qquad\forall\Tilde{X}_\theta\in T_\rho\myPH
\end{equation}
and we can identify the space of exponential representations with the space of mixture representations
\begin{equation}
    T_\rho^{(e)}\myPH\equiv T_\rho^{(m)}\myPH\equiv T_\rho\mathfrak{D}
\end{equation}


\subsection{Quantum Fisher information}
In light of the analogies between the space of quantum states and the space of probability distributions, we can now try to define the Fisher information metric for quantum states. We start from the formulation of the Fisher information metric in terms of the exponential representation of tangent vectors, recalling \cref{eq:fisher-exp-int}
\begin{equation}
    G_F(X,Y)=\mathrm{E}_p[X^{(e)}Y^{(e)}]\quad\forall X,Y\in T_p\mathcal{P}
\end{equation}
Then we can define the quantum Fisher information metric by analogy, symmetrizing the arguments
\begin{align}
    G_{QF}(\Tilde{X}_1,\Tilde{X}_2)&=\mathrm{E}_\rho\left[\frac{1}{2}\left(L^{(\rho)}_1\,L^{(\rho)}_2+L^{(\rho)}_2\,L^{(\rho)}_1\right)\right]\quad\forall \Tilde{X}_1,\Tilde{X}_2\in T_\rho\myPH \notag
    \\
    &=\mathrm{tr}\left(\rho\,\frac{L^{(\rho)}_1\,L^{(\rho)}_2+L^{(\rho)}_2\,L^{(\rho)}_1}{2}\right)\quad \label{eq:qf-exp}
    \\
    &=\mathrm{tr}\left(\frac{1}{2}\left\{\rho,L^{(\rho)}_1\right\}L^{(\rho)}_2\right)=\mathrm{tr}\left(L^{(\rho)}_1\,\frac{1}{2}\left\{\rho,L^{(\rho)}_2\right\}\right)\notag
    \\
    &=\mathrm{tr}\left(d\rho_1\,L^{(\rho)}_2\right)=\mathrm{tr}\left(L^{(\rho)}_1\,d\rho_2\right)
\end{align}
The symmetrization of the arguments is necessary to ensure that the metric is symmetric, and it is also the reason why the choice of the anticommutating term in the SLD has no effect on the geometry. Then, from \cref{eq:qm-exp-mix}, we have that for pure quantum states
\begin{align}
    G_{QF}(\Tilde{X}_1,\Tilde{X}_2)&=2\,\mathrm{tr}\left(d\rho_1\,d\rho_2\right)
    \\
    &=4\,G_{FS}(\Tilde{X}_1,\Tilde{X}_2)
\end{align}
 This shows that the Fubini-Study metric we defined intrinsically for pure quantum states can be precisely interpreted as a quantum information metric.

\subsection{Symmetric covariance}
Our analogy between the geometry of quantum states and the one of probability distributions is now almost complete, but there are still some details to be clarified. Firstly, we would want to treat observables as we treated random variables, but while every tangent vector can be associated with an observable with zero expectation value, the converse is not directly true since
\begin{equation}
    T_\rho\mathfrak{D}\subset\mathfrak{A}^{\bot}_\rho\coloneqq\left\{A\in\mathfrak{A}(\mathcal{H})\mid \mathrm{E}_\rho[A]=0\right\}
\end{equation}
as can be easily checked by comparing the real dimensions of the two vector spaces. Secondly, since operators don't commute in general, we must define a sound generalization of the covariance, and as we will show, these two problems are related.

For starters, we require our generalized covariance to reduce to the variance of the observable when evaluated on the diagonal
\begin{equation}
    \mathrm{Cov}_\rho[A,A]=\mathrm{V}_\rho[A]=\mathrm{tr}\left(\rho\,(A-\mathrm{E}_\rho[A])^2\right)\quad\forall A\in\mathfrak{A}
\end{equation}
Then by requiring it to be symmetric in its arguments, we are naturally led to the definition
\begin{equation}
    \mathrm{Cov}_\rho[A,B]\coloneqq\frac{1}{2}\,\mathrm{tr}\left(\rho\,\left\{A-\mathrm{E}_\rho[A],B-\mathrm{E}_\rho[B]\right\}\right)\quad\forall A,B\in\mathfrak{A}
\end{equation}
Since $(A-\mathrm{E}_\rho[A])$ always has zero expectation value, we can recognize that this expression also defines for every quantum state an inner product in the space of observables with zero expectation value
\begin{equation}
    \llangle A,B \rrangle_\rho\coloneqq\frac{1}{2}\,\mathrm{E}\left[\{A,B\}\right]\quad\forall A,B\in\mathfrak{A}^\bot
\end{equation}
Then we can easily recognize that this inner product coincides with the quantum Fisher metric when the arguments are the exponential representations of tangent vectors
\begin{equation}
    \llangle L^{(\rho)}_1,L^{(\rho)}_2 \rrangle_\rho=G_{QF}(\Tilde{X}_1,\Tilde{X}_2)\quad\forall \Tilde{X}_1,\Tilde{X}_2\in T_\rho\myPH
\end{equation}
This is no coincidence, and it can be shown that the definitions of the exponential representation and of the generalized covariance are deeply linked \cref{amari,matsumoto}.

Let us now go back to the problem of associating every observable with zero expectation value to the exponential representation of a tangent vector. We notice that every observable $A\in\mathfrak{A}$ can be thought of as an element of the Lie algebra of $U(N)$, and thus we can associate it with a tangent vector through the linear map
\begin{equation}
    \mathfrak{M}_\rho:\mathfrak{A}\to T_\rho\mathfrak{D}\quad |\quad A\mapsto\left.\frac{d}{d\theta}\left(e^{iA\theta}\rho e^{-iA\theta}\right)\right\rvert_{\theta=0}=i[A,\rho]\eqqcolon d\rho_A
\end{equation}
The kernel of this map is the linear subspace of the Lie algebra of $U(N)$ that corresponds to the 1-parameter subgroups that leave the quantum state unchanged. We notice that $\mathrm{ker}\,\mathfrak{M}_\rho$ is the set of observables that commute with the pure density operator of the quantum state, i.e, the set of observables for which the state vectors of $\rho$ are eigenvectors; let $\rho=\ket{\psi}\bra{\psi}$ for some normalized state vector $\ket{\psi}$, then
\begin{align}
    [\rho,A]=0&\implies \rho\,A=A\,\rho \notag
    \\
    &\implies A\ket{\psi}=A\,\rho\ket{\psi}=\rho\,A\ket{\psi}=\braket{\psi|A|\psi}\ket{\psi}
    \\
    &\implies A\,\rho=\mathrm{E}_\rho[A]\,\rho
\end{align}
Then any element of $\mathrm{ker}\,\mathfrak{M}_\rho$ can be expressed as
\begin{equation}\label{eq:subgroups}
    K=\lambda^{(k)}\rho + K'\quad\mathrm{with}\quad K'=K'^\dag,\quad K'\,\rho=\rho\,K'=0
\end{equation}
notice that the two components decompose the kernel in two linear subspaces closed under commutation, thus they define two subgroups of $U(N)$
\begin{equation}
    e^{i\theta\lambda^{(k)}\,\rho}\ket{\psi}=e^{i\theta\lambda^{(k)}}\ket{\psi}\qquad e^{i\theta\,K'}\ket{\psi}=e^{i\theta\cdot0}\ket{\psi}=\ket{\psi}
\end{equation}
The first is isomorphic to $U(1)$ and is the subgroup that moves the state vectors of $\rho$ along the fiber. The second one is isomorphic to $U(N-1)$ and leaves the state vectors of $\rho$ unchanged. Through the matrix representation of operators, we can easily compute the real dimension of the lie algebra of $U(N)$ to be $N^2$ for any $N$. Then the dimension of $\mathrm{ker}\,\mathfrak{M}_\rho$ is $1+(N-1)^2$ while the dimension of the domain is $N^2$, so that $T_\rho\mathfrak{D}$ has dimension $2N-2$ as we expected.
\footnote{
    This shows that we could have defined $\mathbb{C}\mathbf{P}^n=\frac{U(N)}{U(1)\times U(N-1)}$. Through the same arguments, this expression can be easily generalized to mixed states, so that the space of rank-m density matrices is given by $\frac{U(N)}{U(m)\times U(N-m)}$
}

We consider now the restriction of $\mathfrak{M}_\rho$ to the observables with zero expectation value in $\rho$
\begin{equation}
    \left.\mathfrak{M}_\rho^{(0)}\right\rvert_{\mathfrak{A}^\bot_\rho}:\mathfrak{A}^\bot_\rho\to T_\rho\mathfrak{D}
\end{equation}
From \cref{eq:subgroups} we can easily compute the real dimensions of the domain and of the kernel of $\mathfrak{M}_\rho^{(0)}$ to be respectively $(N^2-1)$ and $(N-1)^2$. Then the image of $\mathfrak{M}_\rho^{(0)}$ must have a real dimension of $(2N-2)$, equal to the one of the tangent space, thus proving that $\mathfrak{M}_\rho^{(0)}$ is surjective. Then we also have that $\mathfrak{M}_\rho^{(0)}$ maps $T_\rho\mathfrak{D}$ to itself since
\begin{align*}
    [L^{(\rho)},\rho]=0&\implies d\rho=L^{(\rho)}\,\rho=\rho\,L^{(\rho)}
    \\
    &\implies (d\rho-L^{(\rho)})\,\rho=\rho\,(d\rho-L^{(\rho)})
    \\
    &\implies iA\rho-i\rho A\rho -L^{(\rho)}\rho=i\rho A-i\rho A\rho -\rho L^{(\rho)}
    \\
    &\implies i[A,\rho]=[L^{(\rho)},\rho]
    \\
    &\implies d\rho=0,\, L^{(\rho)}=0
\end{align*}
where we considered $d\rho=i[A,\rho]$. It follows that we can decompose the vector space of observables with zero expectation value in $\rho$ as
\begin{equation}
    \mathfrak{A}^\bot_\rho=\mathfrak{A}^\bot_{C(\rho)}\oplus T_\rho\mathfrak{D}\qquad\forall \rho\in\mathfrak{D}
\end{equation}
where $\mathfrak{A}^\bot_{C(\rho)}$ is the set of zero expectation value observables that commute with $\rho$, then we write
\begin{gather}
    A=A_C^{(\rho)}+A^{(\rho)}\qquad\forall A\in\mathfrak{A}^\bot_\rho\notag
    \\
    \mathrm{with}\quad A_C^{(\rho)}\in\mathfrak{A}^\bot_{C(\rho)}\quad\mathrm{and}\quad A^{(\rho)}\in T_\rho\mathfrak{D}
\end{gather}

Finally, we can show that the inner product we defined on $A\in\mathfrak{A}$ depends only on the non-commuting part of the decomposition
\begin{align}
    \llangle A,B \rrangle_\rho&=\frac{1}{2}\,\mathrm{tr}\left(\rho\left\{A,\,B\right\}\right) \notag
    \\
    &=\frac{1}{2}\,\mathrm{tr}\left(\rho\left\{A_C^{(\rho)}+A^{(\rho)},\,B_C^{(\rho)}+B^{(\rho)}\right\}\right) \notag
    \\
    &=\frac{1}{2}\,\mathrm{tr}\left(\rho\left\{A^{(\rho)},\,B^{(\rho)}\right\}\right)+\frac{1}{2}\,\mathrm{tr}\left(\rho\left\{A^{(\rho)},\,B_C^{(\rho)}\right\}\right)+ \notag
    \\
    &\quad+\frac{1}{2}\,\mathrm{tr}\left(\rho\left\{A_C^{(\rho)},\,B^{(\rho)}\right\}\right)+\frac{1}{2}\,\mathrm{tr}\left(\rho\left\{A_C^{(\rho)},\,B_C^{(\rho)}\right\}\right) \notag
    \\
    &=\frac{1}{2}\,\mathrm{tr}\left(\rho\left\{A^{(\rho)},\,B^{(\rho)}\right\}\right)=\llangle A^{(\rho)},B^{(\rho)} \rrangle_\rho
\end{align}
so that we can map every observable with zero expectation value in $\rho$ to a tangent vector of $\rho$ through
\begin{equation}
    A\in\mathfrak{A}^\bot_\rho\mapsto \tilde{A}\in T_\rho\mathbf{P}\mathcal{H}:\quad \tilde{A}^{(e)}=A^{(\rho)}\in T_\rho \mathfrak{D}
\end{equation}
Then we have proved that
\begin{equation}
    \llangle A,B \rrangle_\rho=G_{QF}(\tilde{A},\tilde{B})
\end{equation}
completing our analogy.