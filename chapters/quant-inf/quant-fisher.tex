\section{Quantum information geometry}
\subsection{Observables and the tangent space}

As we have seen in \cref{ch:quantum-geom}, quantum states can be interpreted as a generalization of probability distributions. We now want to further investigate the parallelisms between the geometry of quantum states and the one of probability distributions. 

We start by recalling the representation of tangent vectors to quantum states as linear operators, that we defined in \cref{eq:tgt-operator}
\begin{equation}
    \frac{\partial}{\partial\theta}\in T_\rho\mathfrak{D}\mapsto d\rho_\theta\quad\mathrm{with}\quad d\rho_\theta(\mathbf{X})=\left.\frac{d}{d\theta}\left[\rho(\theta)(\mathbf{X})\right]\right\rvert_{\theta=0}\quad\forall\mathbf{X}\in\mathcal{H}
\end{equation}
This is analogous to the mixture representation of tangent vectors to probability distributions we defined in \cref{eq:mixt-rep}. We then recall from the last section of \cref{ch} that the mixture representation of any tangent vector $d\rho\in T_\rho\mathfrak{D}$ can be expressed as
\begin{equation} \label{eq:QI-mixt-rep}
    d\rho = i[A,\rho]
\end{equation}
for some Hermitian operator $A$. Then it is evident that $d\rho$ is also Hermitian and we may interpret it as an observable; further, we have
\begin{equation}
    \mathrm{tr}(d\rho_\theta)=\frac{d}{d\theta}\left[\mathrm{tr}(\rho(\theta))\right]=0
\end{equation}
that is equivalent to the expression \cref{ff}.

To define an analogous to the exponential representation, we would have to divide the mixture representation by the probability distribution itself but this is not possible since we are dealing with operators. Instead, we can use the implicit expression of \cref{eq:rep-rel} and symmetrize it to obtain
\begin{equation} \label{eq:SLD-mixture}
    d\rho = \frac{1}{2}\left[\rho\,L^{(\rho)}+L^{(\rho)}\,\rho\right]
\end{equation}
where $L^{(\rho)}$ would serve as the analogous to the exponential representation of tangent vectors. To precisely define this operator we express $d\rho$ as in \cref{eq:QI-mixt-rep}, so that we obtain the following equation
\begin{equation}
    i[A,\rho] = \frac{1}{2}\left\{\rho,L^{(\rho)}(A)\right\}
\end{equation}
It can be proved that $L^{(\rho)}(A)$ is well-defined for any operator $A$ when $\rho$ is Hermitian and positive semi-definite\cref{ff}. Then $L^{(\rho)}(A)$ is a linear map between operators and it is known as the \emph{symmetric logarithmic derivative (SLD)} of $A$. Moreover, it can be shown that  $L^{(\rho)}(A)$ is Hermitian if $A$ is Hermitian, so that
\begin{equation}
    L^{(\rho)}:\mathfrak{A}\to\mathfrak{A}\quad\mathrm{with}\quad\mathfrak{A}\coloneqq\left\{A\in\mathfrak{L}(\mathcal{H})\mid A=A^\dag\right\}
\end{equation}
and thus the exponential representations of tangent vectors can also be interpreted as observables. In analogy with \cref{ff} we also have that its expectation value is zero since
\begin{equation}
    \mathrm{E}\left[L^{(\rho)}\right]=\mathrm{tr}(\rho\,L^{(\rho)})=\frac{1}{2}\mathrm{tr}(\rho\,L^{(\rho)}+L^{(\rho)}\,\rho)=\mathrm{tr}(d\rho)=0
\end{equation}

Finally, for pure density operators, we can find an explicit expression for the SLD as follows
\begin{equation}
    d\rho=\frac{d}{d\theta}\left[\rho(\theta)\right]\rvert_{\theta=0}=\frac{d}{d\theta}\left[\rho^2(\theta)\right]\rvert_{\theta=0}=\rho\,d\rho+d\rho\,\rho
\end{equation}
so that for any pure density operator $\rho\in\mathfrak{D}$
\begin{equation}
    L^{(\rho)}_\theta=2d\rho_\theta\qquad\forall d\rho_\theta\in T_\rho\mathfrak{D}
\end{equation}
and since the SLD is a linear map we have that
\begin{equation}
    L^{(\rho)}:\mathfrak{A}\to T_\rho\mathfrak{D}
\end{equation}
is surjective.
We have thus shown that the mixture and exponential representations of tangent vectors to pure quantum states coincide up to a constant factor.
\todo{Talk about the space of the mixture representation}
\todo{Show that the analogy works if quantum states are the pure states?}


\subsection{Quantum Fisher information}

In light of the analogies between the space of quantum states and the space of probability distributions, we can now try to define the analogous of the Fisher information metric for quantum states. We start from the formulation of the Fisher information metric in terms of the exponential representation of tangent vectors, recalling \cref{ref}
\begin{equation}
    G_F(X,Y)=\mathrm{E}_p[X^{(e)}Y^{(e)}]\quad\forall X,Y\in T_p\mathcal{P}
\end{equation}
that is defined in terms of the expectation value of the product of the exponential representations of tangent vectors. Then we can define the quantum Fisher information metric by analogy, symmetrizing the arguments
\begin{align}
    G_{QF}(d\rho_1,d\rho_2)&=\mathrm{E}_\rho\left[\frac{1}{2}\left(L^{(\rho)}_1\,L^{(\rho)}_2+L^{(\rho)}_2\,L^{(\rho)}_1\right)\right]\notag
    \\
    &=\mathrm{tr}\left[\rho\,\frac{L^{(\rho)}_1\,L^{(\rho)}_2+L^{(\rho)}_2\,L^{(\rho)}_1}{2}\right]\quad\forall d\rho_1,d\rho_2\in T_\rho\mathfrak{D}
    \\
    &=\mathrm{tr}\left[\frac{1}{2}\left\{\rho,L^{(\rho)}_1\right\}L^{(\rho)}_2\right]=\mathrm{tr}\left[L^{(\rho)}_1\,\frac{1}{2}\left\{\rho,L^{(\rho)}_2\right\}\right]\notag
    \\
    &=\mathrm{tr}\left[d\rho_1\,L^{(\rho)}_2\right]=\mathrm{tr}\left[L^{(\rho)}_1\,d\rho_2\right]\quad\forall d\rho_1,d\rho_2\in\in T_\rho\mathfrak{D}
\end{align}
so that from \cref{ref} follows that for pure quantum states
\begin{align}
    G_{QF}(d\rho_1,d\rho_2)&=2\,\mathrm{tr}\left[d\rho_1\,d\rho_2\right]\quad\forall d\rho_1,d\rho_2\in T_\rho\mathfrak{D}
    \\
    &=4\,g_{FS}(d\rho_1,d\rho_2)\quad\forall d\rho_1,d\rho_2\in T_\rho\mathfrak{D}
\end{align}
This shows that the Fubini-Study metric we defined intrinsically for pure quantum states can be precisely interpreted as a quantum information metric.

\subsection{Symmetric covariance}
Our analogy between the geometry of quantum states and the one of probability distributions is now almost complete but there are still some details to be precised. For start, we would want to treat observables as we treated random variables but while every tangent vector can be associated with an observable with zero expectation value, the converse is not directly true since
\begin{equation}
    T_\rho\mathfrak{D}\subset\mathfrak{A}^{\bot}_\rho\coloneqq\left\{A\in\mathfrak{A}(\mathcal{H})\mid \mathrm{E}_\rho[A]=0\right\}
\end{equation}
as can be easily checked by comparing the real dimensions of the two vector spaces. Moreover, since operators don't commute in general, we must define a sound generalization of the covariance and as we will show these two problems are related.

For start, we require our generalized covariance to reduce to the variance of the observable when evaluated on the diagonal
\begin{equation}
    \mathrm{Cov}_\rho[A,A]=\mathrm{V}_\rho[A]=\mathrm{tr}\left(\rho\,(A-\mathrm{E}_\rho[A]\mathbb{I})^2\right)\quad\forall A\in\mathfrak{A}
\end{equation}
Then by requiring it to be symmetric in its arguments, we are naturally led to the definition
\begin{equation}
    \mathrm{Cov}_\rho[A,B]\coloneqq\frac{1}{2}\,\mathrm{tr}\left(\rho\,\left\{(A-\mathrm{E}_\rho[A]\mathbb{I}),(B-\mathrm{E}_\rho[B]\mathbb{I})\right\}\right)\quad\forall A,B\in\mathfrak{A}
\end{equation}
Since $(A-\mathrm{E}_\rho[A]\mathbb{I})$ always has zero expectation value, we can recognize that this expression also defines for every quantum state an inner product in the space of observables with zero expectation value
\begin{equation}
    \ll A,B \gg_\rho\coloneqq\mathrm{E}\left[\,\frac{1}{2}\,\{A,B\}\,\right]\quad\forall A,B\in\mathfrak{A}^\bot
\end{equation}
Then we can easily recognize that this inner product coincides with the quantum fisher metric for the exponential representations of tangent vectors of \cref{ref}
\begin{equation}
    \ll L^{(\rho)}_1\,L^{(\rho)}_2 \gg_\rho=G_{QF}(d\rho_1,d\rho_2)\quad\forall d\rho_1,d\rho_2\in T_\rho\mathfrak{D}
\end{equation}
This is no coincidence and it can be shown that the definitions of the exponential representation and of the generalized covariance are deeply linked \cref{amari,matsumoto}.

Let us now go back to the problem of associating every observable with zero expectation value to the exponential representation of a tangent vector. We notice that every observable $A\in\mathfrak{A}$ can be thought of as an element of the Lie algebra of $U(N)$ and thus we can associate it with a tangent vector through the linear map
\begin{equation}
    A\mapsto\left.\frac{d}{d\theta}\left(e^{iA\theta}\rho e^{-iA\theta}\right)\right\rvert_{\theta=0}=i[A,\rho]\eqqcolon d\rho_A
\end{equation}
Then the kernel of this map is the linear subspace of observables that commute with the pure density operator of the quantum state. We can easily interpret this considering the fact that if an observable commutes with the pure density operator of a quantum state, then the state vectors of that quantum state are eigenvectors of the observable. Let $\rho=\ket{\psi}\bra{\psi}$ for some normalized state vector $\ket{\psi}$, then
\begin{align}
    [\rho,A]=0&\implies \rho\,A=A\,\rho \notag
    \\
    &\implies A\ket{\psi}=A\,\rho\ket{\psi}=\rho\,A\ket{\psi}=\braket{\psi|A|\psi}\ket{\psi}
    \\
    &\implies A\,\rho=\mathrm{E}_\rho[A]\,\rho
\end{align}
and so the 1-parameter unitary subgroup of $A$ will move state vectors parallel to the fiber, thus not changing the quantum state.

We can now consider the restriction of this map to the observables with zero expectation value. Through the matrix representation of operators we can easily compute the real dimensions of $\mathfrak{A}^\bot$ and of the kernel of the map. These turn out to be respectively $N^2-1$ and $(N-1)^2$ so that the image of the map must have real dimension $2N-2$, equal to the dimension of the tangent space; this proves that the map is surjective. Then we can also show that the map maps $T_\rho\mathfrak{D}$ to itself since
\begin{align*}
    [L^{(\rho)},\rho]=0&\implies d\rho=L^{(\rho)}\,\rho=\rho\,L^{(\rho)}
    \\
    &\implies (d\rho-L^{(\rho)})\,\rho=\rho\,(d\rho-L^{(\rho)})
    \\
    &\implies iA\rho-i\rho A\rho -L^{(\rho)}\rho=i\rho A-i\rho A\rho -\rho L^{(\rho)}
    \\
    &\implies i[A,\rho]=[L^{(\rho)},\rho]
    \\
    &\implies d\rho=0,\, L^{(\rho)}=0
\end{align*}
where we considered $d\rho=i[A,\rho]$. From this follows that we can decompose the vector space of observables with zero expectation value as
\begin{equation}
    \mathfrak{A}^\bot=\mathfrak{A}^\bot_{C(\rho)}\oplus T_\rho\mathfrak{D}\qquad\forall \rho\in\mathfrak{D}
\end{equation}
where $\mathfrak{A}^\bot_{C(\rho)}$ is the set of zero expectation value observables that commute with $\rho$. Given any observable
with zero expectation value we write
\begin{gather}
    A=A_{C(\rho)}+A^{(e)}_\rho\qquad\forall A\in\mathfrak{A}^\bot\notag
    \\
    \mathrm{with}\quad A_{C(\rho)}\in\mathfrak{A}^\bot_{C(\rho)},\,A^{(e)}_\rho\in T_\rho\mathfrak{D}
\end{gather}
and we thus have that the decomposition is unique.

Finally, we can show that the symmetric covariance we defined only depends on the non-commuting part of the decomposition
\begin{align}
    \ll A,B \gg_\rho&=\frac{1}{2}\,\mathrm{tr}\left[\rho\left\{A,\,B\right\}\right] \notag
    \\
    &=\frac{1}{2}\,\mathrm{tr}\left[\rho\left\{A_{C(\rho)}+A^{(e)}_\rho,\,B_{C(\rho)}+B^{(e)}_\rho\right\}\right] \notag
    \\
    &=\frac{1}{2}\,\mathrm{tr}\left[\rho\left\{A^{(e)}_\rho,\,B^{(e)}_\rho\right\}\right]+\frac{1}{2}\,\mathrm{tr}\left[\rho\left\{A^{(e)}_\rho,\,B_{C(\rho)}\right\}\right]+ \notag
    \\
    &\quad+\frac{1}{2}\,\mathrm{tr}\left[\rho\left\{A_{C(\rho)},\,B^{(e)}_\rho\right\}\right]+\frac{1}{2}\,\mathrm{tr}\left[\rho\left\{A_{C(\rho)},\,B_{C(\rho)}\right\}\right] \notag
    \\
    &=\frac{1}{2}\,\mathrm{tr}\left[\rho\left\{A^{(e)}_\rho,\,B^{(e)}_\rho\right\}\right]=\ll A^{(e)}_\rho,B^{(e)}_\rho \gg_\rho
\end{align}
so that we can define the map
\begin{equation}
    A\mapsto A_\rho\in T_\rho\mathbf{P}\mathcal{H}\qquad\forall A\in\mathfrak{A}^\bot,\,\rho\in\mathbf{P}\mathcal{H}
\end{equation}
where $A_\rho$ is the tangent vector whose exponential rapresenntation is $A^{(e)}_\rho$. Then we have also showed that
\begin{equation}
    \ll A,B \gg_\rho=G_{QF}(A_\rho,B_\rho)
\end{equation}