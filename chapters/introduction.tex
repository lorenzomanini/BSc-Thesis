\chapter*{Introduction}
\addcontentsline{toc}{chapter}{Introduction}

Quantum mechanics is a fundamental theory that provides a mathematical framework for modeling the states and evolution of physical systems, as well as predicting the results of observations. Its development was gradually carried out starting around 1900, in the attempt of explaining physical phenomena that were not understood by classical mechanics, such as the blackbody radiation and the photoelectric effect. By the early-1930s, the theory had been rigorously formulated and many of its predictions had found experimental confirmation.

One of the distinguishing features of quantum mechanics is that the state of a quantum system only determines the probabilities of the outcomes of measurements, and not the outcomes themselves. This hints at the possibility of interpreting quantum states as a generalization of probability distributions and, consequently, of extending the mathematical tools of probability theory to the quantum domain. In this view estimating the state of a quantum system can be seen as a generalization of a parameter estimation problem in probability theory. This is the main idea behind quantum estimation theory (QET) and it will be the subject of this thesis. The treatment that will be followed connects the mathematical formalism of quantum mechanics with the one of probability theory through the respective geometrical descriptions.

The geometrical approach to quantum mechanics was developed starting in the 1970s and has led to many theoretical and experimental advances. In this approach, the state of a quantum system is uniquely represented with a point in a set, called the state space, and the evolutions of the system are described by trajectories in this space. A rich geometric structure arises, the study of which has provided insights into the problem of quantization, the nature of entanglement, and has led to the prediction of later observed phenomena, such as the Berry phase and the Aharonov-Bohm effect.

The study of probability theory from a geometric perspective is known as information geometry and it was developed starting in the 1940s. Probability distributions are represented with points in a set so that assigning some coordinates to a subset is equivalent to considering a family of probability distributions with some parameters. One can then define a distance between probability distributions that measures how easy it is to distinguish them by observing the outcomes. This distance endows the set with a geometric structure, which has been used to study the properties of statistical models and to further develop the theory of parameter estimation. One of the main results of information geometry is the Cramér-Rao bound, which provides a lower bound to how efficiently one can estimate the parameters of a statistical model from the outcomes of measurements.

Starting in the 1990s it was shown that the intrinsic geometry of quantum systems can be understood as a generalization of an information geometric structure. This allowed the development of a quantum version of many results from information geometry and led to many improvements in quantum information and quantum estimation theory. The quantum Cramér-Rao bound is of focal signficance for QET and it is the main result that will be presented in this thesis.

Quantum information and quantum estimation are the theoretical ground of many quantum technologies that are in rapid development in the recent years. Among them QET is especially important for quantum metrology that aims at making high-precision measurements by exploiting the quantum properties of the systems involved.