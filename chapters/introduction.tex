\chapter*{Introduction}
\addcontentsline{toc}{chapter}{Introduction}

Quantum mechanics is a fundamental theory that provides a mathematical framework for modeling the states and evolution of physical systems, as well as predicting the results of observations. Its development began around 1900 to explain physical phenomena that classical mechanics couldn't account for, such as blackbody radiation and the photoelectric effect. By the early 1930s, the theory had been rigorously formulated and many of its predictions had been experimentally confirmed.

One of the distinctive features of quantum mechanics is that the state of a quantum system only determines the probabilities of measurement outcomes, not the outcomes themselves. This suggests the possibility of interpreting quantum states as a generalization of probability distributions and extending the mathematical tools of probability theory to the quantum domain. The two mathematical formalisms can be precisely connected through their geometric descriptions.

The geometrical approach to quantum mechanics emerged in the 1970s and has led to significant theoretical and experimental advancements. In this approach, the state of a quantum system is uniquely represented by a point in a set called the state space, and the system's evolution is described by trajectories in this space. A rich geometric structure arises, and its study has provided insights into quantization, the nature of entanglement, and has predicted phenomena such as the Berry phase and the Aharonov-Bohm effect.

Information geometry, developed in the 1940s, studies probability theory from a geometric perspective. Probability distributions are represented as points in a set, and assigning coordinates to a subset is equivalent to considering a family of probability distributions with parameters. A distance between probability distributions can be defined that measures how easy it is to distinguish them based on observed outcomes. This distance endows the set with a geometric structure, which has been used to study statistical models and develop parameter estimation theory. The Cramér-Rao bound, a key result of information geometry, provides a lower bound on the efficiency of parameter estimation from measurement outcomes.

In the 1990s, it was shown that the intrinsic geometry of quantum systems can be understood as a generalization of an information geometric structure. In this view, estimating the state of a quantum system is a generalization of a parameter estimation problem in probability theory. This approach is known as quantum estimation theory (QET) and the quantum Cramér-Rao bound dictates the best achievable precision of the measurement of a quantum state. Many results from information geometry were generalized in this way to quantum mechanics, bringing significant advances to the theory of quantum information (QI) in general.

Quantum information and quantum estimation theory form the theoretical foundation of various rapidly developing quantum technologies such as quantum computing, quantum metrology, quantum sensing and quantum cryptography. Among them, QET plays a crucial role in quantum metrology, which aims to achieve high-precision measurements by leveraging the quantum properties of the systems involved.

\subsubsection*{Outline}
This thesis is an introductory exposition of the information geometrical structure of quantum systems, with a focus on quantum estimation theory and the quantum Cramér-Rao bound. The treatment is limited to pure states of finite dimensional quantum systems.

In \Cref{part:prob}, we introduce the theory of information geometry and classical parameter estimation. We start by defining the manifold of probability distributions on finite sample spaces and the relation between random variables and representations of tangent vectors. Then we introduce the relative entropy as a pseudo-distance squared and, from it, we induce a Riemannian structure on the manifold of probability distributions: the Fisher information metric. Finally, we use this geometric structure to state and prove the Cramér-Rao bound.

In \Cref{part:quant}, the geometric formulation of quantum mechanics is discussed. We first state the postulates of quantum mechanics in terms of vectors of the Hilbert space, in the finite-dimensional case. Then we define quantum states as equivalence classes of physically undistinguishable state vectors and we relate them to pure density operators, restating the postulates of quantum mechanics in these terms. Finally, we show that the Hilbert space is a fiber bundle with the space of quantum states as base space. With this description, we use the inner product of the Hilbert space to intrinsically define a Riemannian structure on the space of quantum states: the Fubini-Study metric.

In \Cref{part:quant-inf}, we treat the generalization of information geometry to quantum mechanics and we introduce quantum estimation theory. We start by stating several analogies between the geometric structure of quantum states and probability distributions. Then we define the quantum Fisher information generalizing the classic one through the symmetric logarithmic derivative (SLD), and we find that it is equal to the Fubini-Study metric up to a constant factor. Finally, we define quantum estimators and we derive the quantum Cramér-Rao bound.