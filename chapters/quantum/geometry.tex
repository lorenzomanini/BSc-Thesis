\section{The manifold of quantum states}
\subsection{Geometry of the Hilbert space} \label{sec:geometry-HS}
Complex manifolds are defined in analogy with real ones with the requirement of being locally isomorphic to $\mathbb{C}^N$ for some $N$ and with holomorphic transition functions between charts; the tangent space at each point is thus also isomorphic to $\mathbb{C}^N$. Every $N$-dimensional complex manifold is also a $2N$-dimensional real manifold where every complex c
oordinate basis $\{e_1,\dots,e_N\}$ corresponds to a real coordinate basis $\{e_1,\dots,e_N,\,ie_1,\dots,ie_N\}$, this manifold is called the \emph{realification} of the original complex one.

Finite-dimensional Hilbert spaces are isomorphic to $\mathbb{C}^N$ for some $N$ and so they are also trivial complex manifolds. Moreover, there is a canonical isomorphism between the tangent space at each point and the Hilbert space itself
\begin{equation}
    T_{\ket{\psi}}\mathcal{H}\sim\mathcal{H}\qquad\mathrm{with}\qquad\frac{\partial}{\partial\theta} \leftrightarrow \left.\frac{d\ket{\psi(\theta)}}{d\theta}\right\rvert_{\theta=0}\eqqcolon\ket{d\psi_\theta}
\end{equation}
so that the inner product on the Hilbert space defines an inner product on the complex tangent space of every point, i.e., it defines a complex $\binom{0}{2}$ tensor field on the manifold. Hilbert spaces also are metric spaces with respect to the distance induced by the inner product
\begin{equation}
    d(\mathbf{X},\mathbf{Y})=\lVert\mathbf{X}-\mathbf{Y}\rVert=\sqrt{\langle\mathbf{X}-\mathbf{Y},\mathbf{X}-\mathbf{Y}\rangle}
\end{equation}
For finite-dimensional Hilbert spaces, the inner product is a hermitian form that may be expressed as
\begin{equation}
    \langle\mathbf{X},\mathbf{Y}\rangle=h(\mathbf{X},\mathbf{Y})=g(\mathbf{X},\mathbf{Y})+iw(\mathbf{X},\mathbf{Y})
\end{equation}
where $g$ is a real-valued symmetric bilinear form and $w$ is a real-valued antisymmetric bilinear form. Then, the distance may be expressed as
\begin{equation}
    d(\mathbf{X},\mathbf{Y})=\sqrt{g(\mathbf{X}-\mathbf{Y},\mathbf{X}-\mathbf{Y})}
\end{equation}
so that the real part of the inner product also defines a metric tensor on the Hilbert space as in \cref{sec:riemannian}, endowing it with a Riemannian structure
\begin{equation}
    g_{\ket{\psi}}(\ket{d\psi_1},\ket{d\psi_2})=\mathrm{Re}\left[\braket{d\psi_1|d\psi_2}\right]
\end{equation}

This metric tensor is intrinsically derived from the structure of the Hilbert space and so it is invariant under unitary transformations. The set $U(N)$ of unitary transformations of $\mathbb{C}^N$ is a Lie group and thus the Lie algebra of its left-invariant vector fields are the killing vector fields of the Hilbert space. The integral curves of these vector fields also preserve the norm of the points in the Hilbert space, and it can be shown that in each point their tangent vectors form a real $(2N-1)$-dimensional linear subspace of the tangent space. From this follows that the integral curves of the killing vector fields mesh to form a foliation of the realified Hilbert space where each leaf is the set of vectors with a fixed norm and is thus isomorphic to $S^{2N-1}$. Then each leaf is a $(2N-1)$-dimensional submanifold of the realified Hilbert space and inherits a Riemannian structure.

\subsection{Fiber bundle structure of the Hilbert space}
We are now interested in the geometry of the space of quantum states. For start we have that, with a sound choice of coordinates, one can easily verify that $\mathbf{\mathbb{C}P}^N$ is also a complex manifold. \todo{proof?}. Then tangent vectors of $\mathbf{P}\mathcal{H}$ can be intrinsically mapped to operators of $\mathcal{H}$ as follows
\begin{equation}
    \frac{\partial}{\partial\theta}\in T_\rho\mathfrak{D}\mapsto d\rho_\theta\quad\mathrm{with}\quad d\rho_\theta(\mathbf{X})=\left.\frac{d}{d\theta}\left[\rho(\theta)(\mathbf{X})\right]\right\rvert_{\theta=0}\quad\forall\mathbf{X}\in\mathcal{H}
\end{equation}
where from the linearity of differntiation follows that $d\rho_\theta$ is a well-defined linear operator on $\mathcal{H}$.

Now we may ask ourselves if there is a natural way to induce a metric on $\mathbf{P}\mathcal{H}$ from the one of $\mathcal{H}$. Such an intrinsic metric would lead to a natural notion of distance between quantum states that we would then have to interpret. To do this we will investigate the relationship between $\mathcal{H}$ and $\mathbf{P}\mathcal{H}$ as complex manifolds.

Firstly let us now recall the intrinsic projection map of \cref{eq:proj-map-PH}
\begin{equation*}
    \pi:\mathcal{H}_0\to\mathbf{P}\mathcal{H}\quad\ket{\psi}\mapsto[\ket{\psi}]_\sim=\{c\ket{\psi}\,\forall c\ne 0\in\mathbb{C}\}
\end{equation*}
it can be shown that it is a smooth surjective map between the two manifolds so that we may regard $\mathcal{H}_0$ as a fiber bundle with base space $\mathbf{P}\mathcal{H}$. The fibers of the fiber bundle are the orbits of the action on $\mathcal{H}_0$ of the abelian Lie group
\begin{equation}
    \mathbf{C}_0=\{c\cdot\mathbb{I}\mid\forall c\ne0\in\mathbb{C}\}
\end{equation}

Since this group acts smoothly and transitively on the fibers, we have that the fibers are isomorphic to $\mathbf{C}_0$, and the fiber bundle is a principal fiber bundle.

One way to induce a metric on $\mathbf{P}\mathcal{H}$ would be to find a natural embedding of it into $\mathcal{H}_0$. It is a known result from the theory of principal fiber bundles that if such an embedding existed, then the fiber bundle would be the trivial $\mathcal{H}_0=\mathbf{P}\mathcal{H}\times\mathbf{C}_0$. Since this is not the case, there is no natural embedding, i.e., there is no way to choose a representative state vector for every quantum state in terms of the structure of the Hilbert space only.

\subsection{The Fubini-Study metric}

Now that we have described the fiber bundle structure that links $\mathcal{H}_0$ and $\mathbf{P}\mathcal{H}$, we introduce two concepts that will be instrumental in our aim of inducing a metric on the space of quantum states: the push-forward of vectors and the pull-back of 1-forms.

Given a fiber bundle $X$ with base space $B$, the projection map $\pi:X\to B$ allows us to project any smooth curve on the fiber bundle to a smooth curve on the base space. From this follows that we have a natural map
\begin{equation}
    d\pi_x:T_xX\to T_{\pi(x)}B\qquad\forall x\in X
\end{equation}
such that given any smooth curve $\gamma(t)$ with $\gamma(0)=x$, then
\begin{equation}
    d\pi_x(\gamma'(0))=\left.\frac{d}{dt}\left[\pi(\gamma(t))\right]\right\rvert_{t=0}
\end{equation}
where we have used the identification of tangent vectors with equivalence classes of curves. From the linearity of the differentiation follows that $d\pi$ is also linear. Then the push-forward is the map between the tangent bundles of the fiber bundle and the base space such that
\begin{equation}
    d\pi:TX\to TB \qquad\mathrm{with}\qquad d\pi(x,v)=(\pi(x),d\pi_x(v))
\end{equation}
The pull-back of 1-forms is the dual map of the push-forward. Given any 1-form $\alpha\in T^*_{\pi(x)}B$, we define its pull-back on $T^*_xX$ as the 1-form $\alpha'\in T^*_xX$ such that
\begin{equation}
    \alpha'(v)=\alpha(d\pi_x(v))\qquad\forall v\in T_xX
\end{equation}
Then we have that given a 1-form in the cotangent space of some point of the base space, we can pull it back to the cotangent spaces of the entire corresponding fiber of the fiber bundle. From this follows that the pull-back is defined as a map from entire fields of 1-forms in the base space to entire fields of 1-forms in the fiber bundle. This is in contrast with the push forward since for vector fields to be pushed forward, it would be necessary that every vector on the same fiber was pushed to the same vector on the corresponding element of the base space. Finally, using the tensor product, we can similarly define the push-forward of $\binom{p}{0}$ tensors and the pull-back of $\binom{0}{q}$ tensor fields.

The structure we want to project onto $\mathbf{P}\mathcal{H}$ is the $\binom{0}{2}$ metric tensor field, and so we cannot simply push it forward. The fiber bundle we are interested in, $\mathcal{H}_0$, is also a Riemannian principal fiber bundle. As we will see, this allows us to uniquely \emph{lift} vector fields from the base space to the fiber bundle.

Given a fiber bundle $X$ as before, we have that at every point $x\in X$, the curves through $x$ that remain on the same fiber are projected to single points on the base manifold so that the push-forward of their tangent vectors must be the null vector. We define the vertical subspace of $T_xX$ as the kernel of the push-forward
\begin{equation}
    V_x\coloneqq\mathrm{ker}\,d\pi_x\subseteq T_xX
\end{equation}
Then a vector field is vertical if its value at every point is in the vertical subspace. For a general fiber bundle, there may not be any vertical vector fields since the vertical subspaces may not be a smooth subset of its tangent bundle. If $X$ is also a principal fiber bundle with structure group $G$, we have that the fibers are given by the mesh of left-invariant vector fields of $G$. Then the vertical vector fields are precisely the left-invariant vector fields of $G$, and the vertical subspace is isomorphic to the Lie algebra of $G$ at every point
\begin{equation}
    V_x\sim\mathfrak{g}_G
\end{equation}

We may also choose at every point $x\in X$ a horizontal subspace $O_x$, i.e., a complementary subspace to the vertical one so that
\begin{equation}
    T_xX=V_x\oplus O_x
\end{equation}
In general, there is no intrinsic way to choose a horizontal subspace at every point, but if $X$ is also a Riemannian manifold, then we can define the horizontal subspaces as the orthogonal complements of the vertical ones
\begin{equation}
    O_x\coloneqq V_x^\perp
\end{equation} 
and horizontal vector fields are defined in analogy with the vertical ones.

Let us now consider the restriction to $O_x$ of the push-forward at every point
\begin{equation}
    d\pi_x'\equiv d\pi_x\big|_{O_x}:O_x\subseteq T_xX\to T_{\pi(x)}B
\end{equation}
It can be shown that from the smoothness of the metric tensor field follows that $d\pi_x'$ defines a smooth map between the tangent bundles of the fiber bundle and the base space. Then from our previous considerations follows that
\begin{equation}
    \mathrm{ker}\,d\pi_x'=\mathbf{0} \qquad\forall x\in X
\end{equation}
and thus $d\pi_x'$ defines an isomorphism between $O_x$ and $T_{\pi(x)}B$ at every point. We can then uniquely define the \emph{horizontal lift} $v_l(x)\in T_xX$ to the fiber bundle of any vector field $v(y)\in T_yB$ on the base space as
\begin{equation}
    v_l(x)=d\pi_{y}'^{-1}(v(y))\, , \quad y=\pi(x) \qquad\forall x\in X
\end{equation}
that is the unique horizontal vector field on the fiber bundle that projects to $v(y)$.

We can now try to induce a metric tensor field on the base space by identifying its action on vector fields with the action of the metric tensor of the fiber bundle on the horizontal lift of the vector fields of the base space. This is possible only if the scalar field that results from the inner product of the horizontal lifts is constant on the fibers. Only in this case, we can then push forward the scalar field to the base space and thus define the action of the induced metric tensor field on the original vector fields. Requiring the inner product of the horizontal lifts to be constant on the fibers is equivalent to requiring it to be constant under the action of the structure group $G$. By definition, horizontal lifts are invariant under the action of $G$, so the only requirement is that the metric tensor field must be invariant, i.e., the left-invariant vector fields must also be killing vector fields.

We have thus shown that we can induce a metric tensor field on the base space of a Riemannian principal fiber bundle if the elements of the structure group are isometries of the fiber bundle. As we have seen in \cref{sec:geometry-HS}, the set of isometries of $\mathcal{H}_0=\mathbb{C}^N_0$ is the unitary group $U(N)$, while its structure group is the abelian group $\mathbf{C}_0$. Then not every element of the structure group is an isometry since
\begin{equation}
    \mathbf{C}_0=\mathbb{R}^+\times U(1)\qquad c=\lVert c\rVert\cdot e^{i\theta}\leftrightarrow(\lVert c\rVert,e^{i\theta})
\end{equation}
To proceed, we must first fibrate with $\mathbb{R}^+$ and induce a metric on the base of this fibration. This is easily done since we can intrinsically define the embedding of the projection as the normalized state vectors
\begin{equation}
    \ket{\psi}\mapsto\frac{\ket{\psi}}{\lVert\ket{\psi}\rVert}=\frac{\ket{\psi}}{\sqrt{\braket{\psi|\psi}}}\in\mathcal{H}_0\qquad\forall\ket{\psi}\in\mathcal{H}_0
\end{equation}
so that the base space is
\begin{equation}
    S^{2N-1}=\mathbb{C}^N_0/\mathbb{R}^+
\end{equation}
inherits a Riemannian structure such that
\begin{equation}
    g_{\ket{\psi}}(\ket{d\psi_1},\ket{d\psi_2})=\mathrm{Re}\left[\braket{d\psi_1|d\psi_2}\right]
\end{equation}
for any normalized state vector $\ket{\psi}$. Then considering the restriction of the projection map of \cref{eq:proj-map-PH} on $S^{2N-1}$, we can repeat the same arguments so that $S^{2N-1}$ is a Riemannian principal fiber bundle with structure group $U(1)$ and base space
\begin{equation} \label{eq:u-fibr}
    \mathbf{\mathbb{C}P}^n=S^{2N-1}/U(1)
\end{equation}
where $n=N-1$. The structure group is a one-parameter subgroup of the isometry group, so we can induce a metric tensor field on $\mathbf{\mathbb{C}P}^n$ as described in this section. This metric for $\mathbf{\mathbb{C}P}^n$ is known as the Fubini-Study metric $g_{FS}$.

We first derive an explicit expression for the pullback of the Fubini-Study metric on $\mathcal{H}_0$. From our definitions, we have that for any normalized state vector $\ket{\psi}$ 
\begin{equation}
    g_{FS}(\ket{d\psi_1},\ket{d\psi_2})=g(d\pi_\rho'^{-1}(d\rho_1),d\pi_\rho'^{-1}(d\rho_2))
\end{equation}
where $\pi(\ket{\psi})=\rho$ and $d\rho_i=d\pi_{\ket{\psi}}(\ket{d\psi_i})$. We first notice that the linear map
\begin{equation}
    d\pi_\rho'^{-1}\circ d\pi_{\ket{\psi}}:T_{\ket{\psi}}\mathcal{H}\to O_{\ket{\psi}}
\end{equation}
must act as the identity on $O_{\ket{\psi}}$ and must have the kernel $V_{\ket{\psi}}$, the same as $d\pi_{\ket{\psi}}$, so that we can recognize it as the orthogonal projector on $O_{\ket{\psi}}$. Since for every $\ket{\psi}$, the motion along its fiber is given by $c\ket{\psi}$ for some complex number $c$, then
\begin{equation}
    V_{\ket{\psi}}=\{c\ket{\psi}\quad \forall c\in\mathbb{C}\}
\end{equation}
and thus
\begin{equation}
    d\pi_\rho'^{-1}\circ d\pi_{\ket{\psi}}(\ket{d\psi})=(\mathbb{I}-\rho)\ket{d\psi}=\ket{d\psi}-\braket{\psi | d\psi}\ket{\psi}
\end{equation}
so that for normalized state vectors we have
\begin{align}
    g_{FS}(\ket{d\psi_1},\ket{d\psi_2})&=\mathrm{Re}[(\bra{d\psi_2}-\bra{\psi}\braket{d\psi_2| \psi})(\ket{d\psi_1}-\braket{\psi | d\psi_1}\ket{\psi})] \notag
    \\
    &=\mathrm{Re}[\braket{d\psi_2|d\psi_1}-\braket{d\psi_2|\psi}\braket{\psi|d\psi_1}]
\end{align}
Finally then for any state vector $\ket{\psi}\in\mathcal{H}_0$ and two tangent vectors $\ket{d\psi_1},\ket{d\psi_2}\in T_{\ket{\psi}}\mathcal{H}$ we have
\begin{equation}
    g_{FS}(\ket{d\psi_1},\ket{d\psi_2})=\mathrm{Re}\left[\frac{\braket{d\psi_2|d\psi_1}}{\braket{\psi|\psi}}-\frac{\braket{d\psi_2|\psi}\braket{\psi|d\psi_1}}{\braket{\psi|\psi}^2}\right]
\end{equation}

We now want to derive an explicit expression for the Fubini-Study metric on $\mathbf{\mathbb{C}P}^n$. As we have seen the first fibration projects the state vectors to the normalized state vectors, that is a leaf of the foliation of submanifolds of $\mathcal{H}_0$ that are meshed by the integral curves of the killing vector fields, i.e., the left-invariant vector fields of $U(N)$. Then we have that at any point of $S^{2N-1}$ the tangent space is isomorph to the Lie algebra of $U(N)$. The Lie algebra of $U(N)$ is the space of anti-hermitian operators, so that the action of any 1-parameter subgroup of $U(N)$ on a normalized state vector $\ket{\psi}$ can always be expressed as
\begin{equation}
    U(\theta)\ket{\psi}=e^{iA\theta}\ket{\psi}
\end{equation}
for some hermitian operator $A$. Then for any tangent vector $\ket{d\psi}\in T_{\ket{\psi}}S^{2N-1}$ there must exist a unique hermitian operator $A$ such that
\begin{equation}
    \ket{d\psi}=\left.\frac{d}{d\theta}\left(e^{iA\theta}\ket{\psi}\right)\right\rvert_{\theta=0} = iA\ket{\psi}
\end{equation}
The push forwad of this vector is then given by
\begin{align}
    d\rho&=\left.\frac{d}{d\theta}\left(e^{iA\theta}\rho e^{-iA\theta}\right)\right\rvert_{\theta=0}=(iA\rho-\rho iA) \notag
    \\
    &=i[A,\rho]
\end{align}
We can then express the pull-back of the Fubini-Study metric on $S^{2N-1}$ through the push-forwad of the tangent vectors
\begin{align}
    g_{FS}(\ket{d\psi_1},\ket{d\psi_2})&=\mathrm{Re}\left[\braket{d\psi_2|d\psi_1}-\braket{d\psi_2|\psi}\braket{\psi|d\psi_1}\right] \notag
    \\
    &=\mathrm{Re}\left[\braket{\psi|A_2\,A_1|\psi}-\braket{\psi|A_2|\psi}\braket{\psi|A_1|\psi}\right] \notag
    \\
    &=\mathrm{Re}\left[\mathrm{tr}\left(\rho\,(A_2\,A_1-A_2\,\rho\,A_1)\right)\right] \notag
    \\
    &=\frac{1}{2}\mathrm{tr}\left(\rho\,A_2\,A_1+\rho\,A_1\,A_2-\rho\,A_2\,\rho\,A_1-\rho\,A_1\,\rho\,A_2\right) \notag
    \\
    &=\frac{1}{2}\mathrm{tr}\left([A_1,\rho]\,[\rho,A_2]\right)=\frac{1}{2}\mathrm{tr}\left(d\rho_1\,d\rho_2\right)
\end{align}
so that the Fubini-Study metric on $\mathbf{\mathbb{C}P}^n$ is given by
\begin{equation}
    g_{FS}(d\rho_1,d\rho_2)=\frac{1}{2}\mathrm{tr}\left(d\rho_1\,d\rho_2\right)
\end{equation}