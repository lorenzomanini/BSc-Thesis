\section{The manifold of quantum states}
\subsection{The postulates of quantum mechanics} \label{sec:postulates}
Quantum mechanics is a fundamental theory that prescribes a mathematical framework to model the states of physical systems and their evolution, and that describes how to predict the results of our observations of them. We begin by stating its postulates loosely following the treatment of \ref{chuang}.

\begin{postulate}[The Hilbert space]
    Any isolated physical system is associated with a complex \emph{Hilbert space} i.e.\ a, possibly infinite-dimensional, complex vector space $\mathcal{H}$ with an inner product $\langle\cdot,\cdot\rangle$ that is also complete with respect to the metric induced by the inner product.
\end{postulate}

If the Hilbert space is finite-dimensional with dimension $N$, then it is isomorphic to $\mathbb{C}^N$ with a hermitian form as the inner product. In what follows we will only study systems that are associated with a finite-dimensional Hilbert space; the physical meaning of this condition will be discussed later in the section.

With this requirement, the chosen inner product allows us to define a canonical isomorphism between $\mathcal{H}$ and its dual vector space $\mathcal{H}^*$ so that for every $\mathbf{Z}\in\mathcal{H}$ its dual functional is defined as
\begin{equation} \label{eq:duality}
    \mathbf{Z}^*\equiv f_{\mathbf{Z}}:\mathcal{H}\to\mathbb{C} \quad f_{\mathbf{Z}}(\mathbf{X})=\langle\mathbf{Z},\mathbf{X}\rangle \qquad \forall\, \mathbf{X}\in\mathcal{H}
\end{equation}
Then we may also define a tensor $T$ of type $\binom{p}{q}$ as a multilinear map
\begin{equation}
    T:{\mathcal{H}^*}^p\times\mathcal{H}^q \to \mathbb{C}
\end{equation}
where from \cref{eq:duality} follows that vectors can be identified with tensors of type $\binom{1}{0}$ and dual vectors with tensors of type $\binom{0}{1}$. We also have that operators (linear maps from $\mathcal{H}$ to itself) can be identified with $\binom{1}{1}$ tensors since for any operator $A$ we can define the tensor
\begin{equation}
    T_A(u^*,v)=u^*(Av)=\braket{u,Av}
\end{equation}
Then we may define the tensor product $\otimes$ between any two tensors $T$ and $T'$ of type $\binom{p}{q}$ and $\binom{p'}{q'}$ as the tensor of type $\binom{p+p'}{q+q'}$ such that
\begin{align}
    &(T\otimes T')(u^*_1,\dots,u^*_{p+p'},v_1,\dots,v_{q+q'})= \notag
    \\
    &T(u^*_1,\dots,u^*_{p},v_1,\dots,v_{q})\cdot T'(u^*_{p+1},\dots,u^*_{p+p'},v_{q+1},\dots,v_{q+q'})
\end{align}
and, in particular, given a vector $v$ and a dual vector $u^*$ their tensor product is the operator
\begin{equation} \label{eq:outer-prod}
    A(w)=u^*(w)\cdot v=\braket{u,w}\cdot v
\end{equation}

Given any operator, we can also define its adjoint as the operator $\mathrm{A}^\dag$ such that
\begin{equation} \label{eq:adjoint}
    \langle\mathrm{A}^\dag\,\mathbf{X},\mathbf{Y}\rangle=\langle\mathbf{X},\mathrm{A}\,\mathbf{Y}\rangle \qquad \forall\, \mathbf{X},\mathbf{Y}\in\mathcal{H}
\end{equation}
it can be shown that this is always well-defined for finite-dimensional Hilbert spaces and that $(\mathrm{A}^\dag)^\dag=\mathrm{A}$. There are two families of operators that will be instrumental to the formulation of the remaining postulates: unitary operators and self-adjoint operators. Unitary operators are defined as operators U that preserve the inner product, i.e.
\begin{equation}
    \braket{\mathrm{U}\mathbf{X},\mathrm{U}\mathbf{Y}}=\braket{\mathbf{X},\mathbf{Y}}\qquad \forall \mathbf{X},\mathbf{Y}\in\mathcal{H}
\end{equation}
From \cref{eq:adjoint} it's easy to show that
\begin{equation}
    \textit{U is unitary} \iff \mathrm{U}\mathrm{U}^\dag=\mathbb{I}\textit{ i.e }\mathrm{U}^{-1}=\mathrm{U}^\dag
\end{equation}
from which also follows $\mathrm{U}\mathrm{U}^\dag=\mathrm{U}^\dag\mathrm{U}$. Self-adjoint operators are defined as operators A that are equal to their adjoint, i.e.
\begin{equation} \label{eq:selfadj}
    \langle\mathrm{A}\,\mathbf{X},\mathbf{Y}\rangle=\langle\mathbf{X},\mathrm{A}\,\mathbf{Y}\rangle \qquad \forall\, \mathbf{X},\mathbf{Y}\in\mathcal{H}
\end{equation}
or, equivalently, $\mathrm{A}=\mathrm{A}^\dag$. From their definition follows immediately that
\begin{equation} \label{eq:selfadj-real}
    \langle\mathrm{A}\,\mathbf{X},\mathbf{X}\rangle=\overline{\langle\mathbf{X},\mathrm{A}\,\mathbf{X}\rangle}=\overline{\langle\mathrm{A}\,\mathbf{X},\mathbf{X}\rangle}\in\mathbb{R}\qquad \forall\, \mathbf{X}\in\mathcal{H}
\end{equation}

We can now state the remaining postulates of quantum mechanics for the finite-dimensional case.
\begin{postulate}[The state vectors] \label{pt:state-vect}
    Every non-zero vector of the Hilbert space completely characterizes a possible state of the system, we call such vectors \emph{state vectors}. The state vectors of the Hilbert space describe all the possible states of the system.
\end{postulate}
For state vectors, we will also use the Dirac notation writing vectors as $\ket{\psi}$ and their dual as $\bra{\psi}$. An exhaustive treatment of this notation may be found in \cref.

\begin{postulate}[Unitary evolution] \label{pt:unit-evol}
    The state vectors of a closed system evolve only through unitary transformations of the Hilbert space. That is, the time evolution of any state vector $\ket{\psi(t)}$ is given by
    \begin{equation}
        \ket{\psi(t_2)}=\mathrm{U(t_1,t_2)}\ket{\psi(t_1)}
    \end{equation}
    where $U(t_1,t_2)$ is a unitary operator that only depends on $t_1$ and $t_2$.
\end{postulate}
We may interpret this as requiring that the evolution of a closed system preserves the structure we defined on the set $\mathcal{H}$, that is the vector space structure and the inner product space structure. Thus we expect the transformations to be invertible, and linear and to preserve the inner product; in this sense, unitary operators are the automorphisms of the Hilbert space.
\begin{postulate}[Quantum measurements] \label{pt:quant-meas}
    Quantum measurements are described by a collection of pairs $\{(\mathrm{M}_x,x)\}$ of \emph{measurement operators} $\mathrm{M}_x$ and \emph{outcomes} $x$ such that the following \emph{completeness equation} is satisfied
    \begin{equation} \label{eq:completeness}
        \sum_{x}\mathrm{M}_x^\dag\,\mathrm{M}_x=\mathbb{I}
    \end{equation}
    where $\mathbb{I}$ is the identity operator. Then, given a system in a state described by a state vector $\ket{\psi}$, the probability distribution of the outcomes is
    \begin{equation} \label{eq:quant-probability}
        p(x)=\frac{\braket{\psi|\mathrm{M}_x^\dag\,\mathrm{M}_x|\psi}}{\braket{\psi|\psi}}
    \end{equation}
    Finally, any interaction with the system that leads to the measurement of a specific outcome $x$ transforms any state vector $\ket{\psi}$ before the measurement to a new state vector $\ket{\psi'}$ after the measurement according to
    \begin{equation}
        \ket{\psi'}=\mathrm{M}_x\ket{\psi}
    \end{equation}
    that depends on the outcome measured.
\end{postulate}
The probabilities of \cref{eq:quant-probability} are well defined since
\begin{align}
    p(x)&=\frac{\left\lVert\mathrm{M}_x\ket{\psi}\right\rVert^2}{\left\lVert\ket{\psi}\right\rVert^2}\ge 0 \qquad \forall x
    \\
    \sum_{x}p(x)&=\frac{\braket{\psi|\sum_{x}\mathrm{M}_x^\dag\,\mathrm{M}_x|\psi}}{\braket{\psi|\psi}}=1
\end{align}
We thus have that for any fixed measurement every state vector defines a probability distribution on the outcomes. In this sense, quantum states can be thought of as a generalization of probability distributions.
\todo{Example}
\todo{Composed systems and tensor product}

\subsection{PVM and observables}
There is a special class of quantum measurements we will be interested in: projection-valued measurements (PVM). PVMs are quantum measurements where the measurement operators $\{P_x\}$ are required to be orthogonal projectors, i.e.
\begin{align}
    P_x^\dag&=P_x \qquad \forall x
    \\
    P_x^2&=P_x \qquad \forall x
\end{align}
and to form a complete set of orthogonal projectors
\begin{align}
    P_x\,P_y&=\delta_{xy}P_x \qquad \forall x,y \label{eq:ortnorm-pro}
    \\
    \sum_{x}P_x&=\mathbb{I} \label{eq:complete-pro}
\end{align}
To understand these definitions we first state an important theorem from linear algebra about self-adjoint operators, a complete treatment and proof can be found in \cref.
\begin{theorem}[Spectral theorem] \label{th:spectral}
    Let $\mathcal{H}$ be a finite-dimensional complex Hilbert space and $\mathrm{A}$ a self-adjoint operator on $\mathcal{H}$. Then there exists an orthonormal basis of eigenvectors of $\mathrm{A}$ with real eigenvalues.
\end{theorem}
This means that for any self-adjoint operator, we may decompose the Hilbert space in orthogonal linear subspaces of eigenvectors with the same eigenvalue, the \emph{eigenspaces}.

For any orthogonal projector $P$, we can show that
\begin{align}
    P\ket{\psi}=\lambda\ket{\psi} &\implies P^2\ket{\psi}=\lambda^2\ket{\psi}=P\ket{\psi}=\lambda\ket{\psi} \notag
    \\
    &\implies \lambda^2=\lambda \notag
    \\
    &\implies \lambda=0,1
\end{align}
and, given an orthonormal base of eigenvectors $\{\ket{e_i}\}$ we may partition it in the two subsets $\{\ket{e_j^{(0)}}\}$ and $\{\ket{e_k^{(1)}}\}$ respectivly of eigenvectors with eigenvalue 0 and 1. Then by expressing a generic vector as a linear combination of this base, we have that
\begin{equation} \label{eq:proj-decomp}
    \ket{\psi}=\sum_{i}\braket{e_i|\psi}\ket{e_i} \implies P\ket{\psi}=\sum_{k}\braket{e_k^{(1)}|\psi}\ket{e_k^{(1)}}
\end{equation}
so that any orthogonal projector "orthogonally projects" vectors to its eigenspace with eigenvalue 1, which is thus also its image.

We can now recognize that a set of orthonormal projectors is complete when the spaces on which they project are orthogonal (\cref{eq:ortnorm-pro}) and add up to all the Hilbert space (\cref{eq:complete-pro}). With this setting, it can be easily shown that the following corollary holds
\begin{corollary} \label{cr:spect-decomp}
    Any self-adjoint operator $\mathrm{A}$ with eigenvalues $\{\lambda_i\}$ may be expressed in term of the projectors $P_i$ of its eigenspaces as
    \begin{equation} \label{eq:spect-decomp}
        \mathrm{A}=\sum_{i}\lambda_i\,P_i
    \end{equation}
\end{corollary}

We thus have that, intuitively, PVMs are defined by decomposing the Hilbert space in orthogonal subspaces and then assigning a certain outcome to each one. In fact, if $\ket{\psi_x}$ is in the projected space of $P_x$ the probability distribution of the outcomes will be
\begin{equation}
    p(y)=\frac{\braket{\psi_x|P_y^\dag\,P_y|\psi_x}}{\braket{\psi_x|\psi_x}}=\delta_{xy}
\end{equation}
then, after the measurement, the state vector is projected on the subspace of the outcome
\begin{equation}
    \ket{\psi'}=P_x \ket{\psi}
\end{equation}
With this interpretation it's easy to recognize an interesting property of PVMs: repeating the same PVM multiple times while the Hilbert space is evolving with the identity operator (i.e. is not changing) always leads to identical results. In fact, after the first measurement, if the measured outcome was $x$, we will have that
\begin{equation}
    \ket{\psi'_{(0)}}=P_x \ket{\psi}
\end{equation}
for any initial vector state $\ket{\psi}$. Then, repeating the same measurement we will have the probability distribution
\begin{equation}
    p(y)=\frac{\braket{\psi|P_x^\dag\,P_y^\dag\,P_y\,P_x|\psi}}{\braket{\psi|P_x^\dag\,P_x|\psi}}=\delta_{xy}
\end{equation}
and so we will get with certainty the same result. After the measurement, we will have the state vector
\begin{equation}
    \ket{\psi'_{(1)}}=P_x P_x \ket{\psi}=P_x \ket{\psi}
\end{equation}
and so we can reiterate the same argument for the following measurements.

When we think of well-defined measurable properties of a system we may require repeated measurements to always give the same outcome if in between them the system was unchanged. From this intuitive concept follows the definition of an \emph{observable} as a PVM with real-valued outcomes. Then, from \cref{cr:spect-decomp} follows that there is a 1-1 relationship between observables and selfadjoint operators through \cref{eq:spect-decomp} so that we may identify any observable with its selfadjoint operator.
\todo{expectation value?}

\subsection{Quantum states}
Following \crefrange{pt:state-vect}{pt:quant-meas} we have that if we are given a state vector for the system we will know how it will evolve under unitary evolutions, the probability distributions of the outcomes of quantum measurements, and the vector state we will get after those measurements, depending on the outcomes. Then we may ask ourselves if there are multiple state vectors that for any measurement yield the same probabilities, and that continue to do so after any unitary evolution or measurement. Such two vectors would be completely equivalent in their predictions and so we may regard them as describing the same quantum state. What we have just described is an equivalence relation between state vectors so that quantum states are the equivalence classes.

We start by requiring that two equivalent state vectors have the same probability distributions for any measurement. This means that
\begin{equation}
    \ket{\psi}\sim\ket{\psi'} \iff \frac{\braket{\psi|\mathrm{M}_x^\dag\,\mathrm{M}_x|\psi}}{\braket{\psi|\psi}}=\frac{\braket{\psi'|\mathrm{M}_x^\dag\,\mathrm{M}_x|\psi'}}{\braket{\psi'|\psi'}} \qquad \forall x
\end{equation}
for any quantum measurement $\{(M_x,x)\}$. It's easy to guess that a sufficient condition is
\begin{equation}\label{eq:equiv-state-vect}
    \exists\,c \ne 0 \in\mathbb{C} :\quad \ket{\psi'}=c\ket{\psi}
\end{equation}
since for any operator $\mathrm{M}_x$
\begin{equation}
    \frac{\braket{\psi'|\mathrm{M}_x^\dag\,\mathrm{M}_x|\psi'}}{\braket{\psi'|\psi'}}=\frac{\lVert c \rVert^2}{\lVert c \rVert^2}\cdot\frac{\braket{\psi|\mathrm{M}_x^\dag\,\mathrm{M}_x|\psi}}{\braket{\psi|\psi}}
\end{equation}
Then by choosing $\mathrm{M}_x=\ket{\psi}\bra{\psi}$ it's easy to show that \cref{eq:equiv-state-vect} is also a necessary condition.

We also note that for any operator $A$
\begin{equation}
    A(c\ket{\psi})=cA\ket{\psi} \quad \forall c\in\mathbb{C}
\end{equation}
so that any operator has a well-defined action on equivalence classes. In particular, this holds for unitary operators and measurement operators, and so two state vectors in the same equivalence class will remain in the equivalence class after any unitary evolution or measurement.

We have thus proved that quantum states are the equivalence classes of
\begin{equation}
    \ket{\psi}\sim\ket{\psi'} \iff \exists\,c \ne 0 \in\mathbb{C} :\quad \ket{\psi'}=c\ket{\psi}
\end{equation}
Then for any state vector $\ket{\psi}$ its quantum state is
\begin{equation}
    \left[\ket{\psi}\right]_\sim=\{c\ket{\psi}\,\forall c\in\mathbb{C}\}\setminus\{0\}
\end{equation}
and so the space of quantum states is isomorphic to the set of 1-dim linear subspaces of $\mathcal{H}$, also called \emph{complex lines}. When $\mathcal{H}=\mathbb{C}^N$ this set is known as the $N$-dimensional complex projective space $\mathbf{\mathbb{C}P}^N$ and it will be the object of our study.

\subsection{Density matrices}
As explained in the previous subsection quantum states are the 1-dim linear subspaces of $\mathcal{H}$, thus we should be able to identify them with orthogonal projectors. The only added requirement is that the projected spaces must be 1-dim, i.e. they must be rank-1 orthogonal projectors. From \cref{eq:proj-decomp} and \cref{eq:outer-prod} we have that for any rank-1 projector
\begin{equation}
    P_{\tilde{e}}\ket{\psi}=\braket{\tilde{e}|\psi}\ket{\tilde{e}} \implies  P_{\tilde{e}}=\ket{\tilde{e}}\bra{\tilde{e}}
\end{equation}
for some normalized vector $\ket{\tilde{e}}$. 

