\chapter{Geometry of probability distributions}
Quantum states, as we will see, can be thought of as a generalization of probability distributions. In this chapter, we will study probability distributions from a geometrical point of view, and in this framework, we shall prove the Cramer-Rao bound. We will only consider probability distributions defined on finite sample spaces since this is all we need for finite-dimensional pure quantum states; nevertheless, in \ref{} generalization to countable sample spaces will be discussed briefly.

\section{Space of probability distributions}
Consider a \emph{random variable} that can take values in a finite set $\mathcal{X}$ of cardinality $N$ that we call \emph{sample space}. Then a \emph{probability distribution} on $\mathcal{X}$ is a function $p\colon\mathcal{X}\to\mathbb{R}$ which satisfies
\begin{equation}
    p(x)\ge 0 \quad \forall x\in\mathcal{X} \quad \text{and} \quad \sum_{x\in\mathcal{X}}p(x)=1
\end{equation}
where $p(x)$ is the probability that the random variable is found with value $x$. Accordingly, the \emph{space of probability distributions} is
\begin{equation}
    \mathcal{P} = \Bigl\{ p\colon\mathcal{X}\to\mathbb{R} \bigm\vert p(x)\ge 0 \quad \text{and} \quad \sum_{x\in\mathcal{X}}p(x)=1 \Bigr\}
\end{equation}
that we consider equipped with the point-wise topology.
\todo{Decidere se parlare della topologia}

Since we are working with a finite sample space we can consider the isomorphism between functions $f\colon\mathcal{X}\to\mathbb{R}$ and $\mathbb{R}^N$ through 
\begin{equation*}
    f\leftrightarrow\boldsymbol{f}\coloneqq(f(x_1),f(x_2),\dots,f(x_N))
\end{equation*}
so that
\begin{equation}
    \mathcal{P}\sim\Bigl\{ \boldsymbol{p}\in\mathbb{R}^N \bigm\vert p_i\geq 0  \quad \text{and} \quad \sum p_i=1 \Bigr\}
\end{equation}
where the standard topology induced by this isomorphism corresponds to the point-wise one.
\todo{Decidere se parlare della topologia}

We can conclude that the space of probability distributions is the (N-1)-simplex generated by convex mixing of the trivial distributions, as seen in \ref{}; in \ref{} pictures for the first few dimensions are given.
\todo{Decidere se parlare degli spazi convessi}

\subsection*{Examples}
\todo{Esempi}

\section{Statistical manifolds}
We call a n-dimensional \emph{statistical model} on $\mathcal{X}$ a family of probability distributions that are globally parametrized by n real-valued variables. Formally this is a set $\mathcal{S}\subseteq\mathcal{P}$ with an invertible function $\psi\colon\mathcal{S}\to\Xi\subseteq\mathbb{R}^n$, so that we may write
\begin{equation}
    \mathcal{S} = \Bigl\{ p_\xi\in\mathcal{X} \bigm\vert \exists\xi=[\xi^1,\xi^2,\dots,\xi^n]\in\Xi \colon p_\xi=\psi^{-1}(\xi) \Bigr\}
\end{equation}
where $p_\xi(x)$ may be equivalently written as $p(x;\xi)$ or $p(x;\xi^1,\xi^2,\dots,\xi^n)$.\todo{Esempio} We also require two important additional properties:
\begin{itemize}
    \item we assume $\Xi$ is an open set
    \item we assume $\psi^{-1}$ is $\mathit{C}^\infty$
\end{itemize}
This allows us to differentiate the probability distributions with respect to the parameters so that $\partial_ip(x;\xi)$ is well defined, where we wrote $\partial_i\coloneqq\frac{\partial}{\partial\xi_i}$. 

These conditions also imply that $\mathcal{S}$ is an open subset of $\mathcal{P}$ and so the pair $\mathcal{S}$ and $\psi$ form a chart. Then by taking parametrizations that are $\mathit{C}^\infty$ diffeomorphic to each other, we can construct manifolds as described in \ref{}, where statistical models are the charts and the different parametrizations are the coordinate systems. We call such a manifold a \emph{statistical manifold}.

\subsection*{Examples}
\todo{Esempi}

\section{The Fisher metric}