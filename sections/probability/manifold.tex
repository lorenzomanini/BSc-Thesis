\chapter{Manifolds of probability distributions}
\section{Space of probability distributions1}
%Define sample space X as just the space of possible outcomes
%Define right away R^X
%Define probability distributions saying they also are in R^X
%Talk about R^X being a convex space and how one could picture R^X in his mind thinking of R^N
%Talk about the expectation value and the Euclidean product between rnd var and p dist
Consider a \emph{random process} and the set $\mathcal{X}$ of its possible outcomes. We call this set the \emph{sample space}, and we will only consider random processes for which it is finite. Then, a \emph{probability distribution} on $\mathcal{X}$ is a function $p\in\mathbb{R}^\mathcal{X}\coloneq\{f\mid f\colon \mathcal{X}\to\mathbb{R}\}$ which satisfies
\begin{equation} \label{eq:prob-dist-def}
    p(x)\ge 0 \quad \forall x\in\mathcal{X} \quad \text{and} \quad \sum_{x\in\mathcal{X}}p(x)=1
\end{equation}
where $p(x)$ represents the probability of the outcome $x$. We also have that every function $A\in\mathbb{R}^\mathcal{X}$ represents a real \emph{random variable}, as it maps every outcome of a random process to a number.

Let now $N$ be the cardinality of $\mathcal{X}$, then we can consider the natural isomorphism between $\mathbb{R}^\mathcal{X}$ and $\mathbb{R}^N$ by indexing the outcomes
\begin{equation}
    f \leftrightarrow (f(x_1),\dots,f(x_N))
\end{equation}
then it's easy to see that the \emph{space of probability distributions} is the convex set resulting from the convex mixing of the trivial probability distributions $f(x_i)=\delta_{ik}$, represented by the unit vectors of $\mathbb{R}^N$.

It's also interesting to consider the inner product induced on $\mathbb{R}^\mathcal{X}$ by the Euclidean one of $\mathbb{R}^N$. In particular, let $p$ be a probability distribution and $A$ a random variable, then
\begin{equation}
    p \cdot A = \sum_{x\in\mathcal{X}}p(x)A(x)\coloneqq\mathrm{E}_p[A]
\end{equation}
where $\mathrm{E}_p[A]$ is the \emph{expectation value} of the random variable $A$ when the underlying probability distribution of the random process is $p$. 

\section{Space of probability distributions2}
Consider a \emph{random variable} that can take values in a finite set $\mathcal{X}$ of cardinality $N$ that we call \emph{sample space}. Then a \emph{probability distribution} on $\mathcal{X}$ is a function $p\colon\mathcal{X}\to\mathbb{R}$ which satisfies
\begin{equation}
    p(x)\ge 0 \quad \forall x\in\mathcal{X} \quad \text{and} \quad \sum_{x\in\mathcal{X}}p(x)=1
\end{equation}
where $p(x)$ represents the probability that the random variable is found with value $x$. Accordingly, the \emph{space of probability distributions} is
\begin{equation}
    \mathcal{P} = \Bigl\{ p\colon\mathcal{X}\to\mathbb{R} \bigm\vert p(x)\ge 0 \quad \text{and} \quad \sum_{x\in\mathcal{X}}p(x)=1 \Bigr\}
\end{equation}
that we consider equipped with the point-wise topology.
\todo{Decidere se parlare della topologia}

Since we are working with a finite sample space we can consider the isomorphism between functions $f\colon\mathcal{X}\to\mathbb{R}$ and $\mathbb{R}^{card(\mathcal{X})}=\mathbb{R}^N$ through 
\begin{equation*}
    f\leftrightarrow\boldsymbol{f}\coloneqq(f(x_1),f(x_2),\dots,f(x_N))
\end{equation*}
so that
\begin{equation} \label{eq:P-conv-emb}
    \mathcal{P}\sim\Bigl\{ \boldsymbol{p}\in\mathbb{R}^N \bigm\vert p_i\geq 0  \quad \text{and} \quad \sum p_i=1 \Bigr\}
\end{equation}
where the standard topology induced by this isomorphism corresponds to the point-wise one. We call this the convex embedding of $\mathcal{P}$ in $\mathbb{R}^N$.
\todo{Decidere se parlare della topologia}

We can conclude that the space of probability distributions is the (N-1)-simplex generated by convex mixing of the trivial distributions, as seen in \ref{}; in \ref{} pictures for the first few dimensions are given.
\todo{Decidere se parlare degli spazi convessi}

\todo{Esempi}

\section{Statistical models and manifolds}
We call an n-dimensional \emph{statistical model} on $\mathcal{X}$ a family of probability distributions that are globally parametrized by n real-valued variables. Formally this is a set $\mathcal{S}\subseteq\mathcal{P}$ with an invertible function $\psi\colon\mathcal{S}\to\Xi\subseteq\mathbb{R}^n$, so that we may write
\begin{equation}
    \mathcal{S} = \Bigl\{ p_\xi\in\mathcal{P} \bigm\vert \exists\xi=[\xi^1,\xi^2,\dots,\xi^n]\in\Xi \colon p_\xi=\psi^{-1}(\xi) \Bigr\}
\end{equation}
where $p_\xi(x)$ may be equivalently written as $p(x;\xi)$ or $p(x;\xi^1,\xi^2,\dots,\xi^n)$. This definition of a statistical model reflects the act of hypothesizing an underlying model, that may depend on some parameters, for the generation of the random variable's samples. Then only a subset, here represented by $\mathcal{S}$, of all the possible probability distributions is considered as a candidate of the underlying probability distribution, and every candidate probability distribution is identified uniquely by the corresponding parameters, here represented by $\xi$.

We also require two important additional regularity properties:
\begin{equation}
    \text{$\mathcal{S}$ is an open set}\qquad\text{and}\qquad\text{$\psi^{-1}$ is $\mathit{C}^\infty$}
\end{equation}
that immediately imply that $\Xi$ is also an open set. This allows us to differentiate the probability distributions with respect to the parameters so that $\partial_ip(x;\xi)$ is well defined, where we wrote $\partial_i\coloneqq\frac{\partial}{\partial\xi_i}$. These conditions also imply that the pair $\mathcal{S}$ and $\psi$ form a chart (both of $\mathcal{P}$ and $\mathcal{S}$). Then by taking parametrizations that are $\mathit{C}^\infty$ diffeomorphic to each other, we can construct $\mathit{C}^\infty$ manifolds as described in \ref{}, where statistical models are the charts and the different parametrizations are the coordinate systems; we call manifolds like these \emph{statistical manifolds}.

\todo{Esempi}
