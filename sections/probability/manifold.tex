\chapter{Manifolds of probability distributions}
\section{Space of probability distributions} \label{ch:space-prob-dist}
Consider a \emph{random process} and the set $\mathcal{X}$ of all its possible outcomes. We call this set the \emph{sample space}, and we will only consider random processes for which it is finite. Then, a \emph{probability distribution} on $\mathcal{X}$ is a function $p\in\mathbb{R}^\mathcal{X}\coloneq\{f\mid f\colon \mathcal{X}\to\mathbb{R}\}$ which satisfies
\begin{equation} \label{eq:prob-dist-def}
    p(x)\ge 0 \quad \forall x\in\mathcal{X} \quad \text{and} \quad \sum_{x\in\mathcal{X}}p(x)=1
\end{equation}
where $p(x)$ represents the probability of the outcome $x$. We also have that every function $A\in\mathbb{R}^\mathcal{X}$ represents a real \emph{random variable}, as it maps every outcome of a random process to a number.

Let now $N$ be the cardinality of $\mathcal{X}$. To have a picture of $\mathbb{R}^\mathcal{X}$ we can index the outcomes and consider the natural isomorphism between $\mathbb{R}^\mathcal{X}$ and $\mathbb{R}^N$
\begin{equation}
    f \leftrightarrow (f(x_1),\dots,f(x_N))
\end{equation}
then it's easy to recognize that the \emph{space of probability distributions} is a convex subset of the affine subspace $\mathcal{A}_1\coloneqq\{f\in\mathbb{R}^\mathcal{X}\mid\sum_{x\in\mathcal{X}}f(x)=1\}$. In particular, it is the set resulting from the convex mixing of the trivial probability distributions $f(x_i)=\delta_{ik}$, represented by the unit vectors of $\mathbb{R}^N$.

It's also interesting to consider the inner product induced on $\mathbb{R}^\mathcal{X}$ by the Euclidean one of $\mathbb{R}^N$. Let $p$ be a probability distribution and $A$ a random variable, then
\begin{equation}
    p \cdot A = \sum_{x\in\mathcal{X}}p(x)A(x)\coloneqq\mathrm{E}_p[A]
\end{equation}
where $\mathrm{E}_p[A]$ is the \emph{expectation value} of the random variable $A$ when the underlying probability distribution of the random process is $p$.

\section{Statistical models and manifolds}
We call an n-dimensional \emph{statistical model} on $\mathcal{X}$ a family of probability distributions that are globally parametrized by n real-valued variables. Formally this is a subset $\mathcal{S}$ of the space of probability distributions with an invertible function $\psi\colon\mathcal{S}\to\Xi\subseteq\mathbb{R}^n$, so that we may write
\begin{equation}
    \mathcal{S} = \Bigl\{ p_\xi \bigm\vert \exists\,\xi=(\xi^{(1)},\dots,\xi^{(n)})\in\Xi \colon p_\xi=\psi^{-1}(\xi) \Bigr\}
\end{equation}
where $p_\xi(x)$ may be equivalently written as $p(x;\xi)$ or $p(x;\xi^{(1)},\dots,\xi^{(n)})$. This definition of a statistical model reflects the act of hypothesizing an underlying model, that may depend on some parameters, for the generation of the random variable's samples. Then only a subset, here represented by $\mathcal{S}$, of all the possible probability distributions is considered as a candidate of the underlying probability distribution, and every candidate probability distribution is identified uniquely by the corresponding parameters, here represented by $\xi$.

We now introduce some additional requirements to $S$ so that we may treat it as a well-behaved manifold. Firstly we regard $S$ as a subset of $\mathcal{A}_1$ equipped with the topology induced by the standard one of $\mathbb{R}^N$. Then, we require that
\begin{equation} \label{eq:param-prop}
    \begin{aligned}
        &\text{$\Xi$ is an open set}\\
        &\text{$\psi$ is a $\mathit{C}^\infty$ diffeomorphism from $\mathcal{S}$ to $\Xi$}
    \end{aligned}
\end{equation}
This allows us to differentiate the probability distributions with respect to the parameters so that $\partial_ip(x;\xi)$ is well defined, where we wrote $\partial_i\coloneqq\frac{\partial}{\partial\xi^{(i)}}$. These conditions also imply that the pair $\mathcal{S}$ and $\psi$ form a chart of $\mathcal{S}$. Then for any another statistical model on $\mathcal{S}$ with parametrization $\psi'\colon\mathcal{S}\to\Xi'\subseteq\mathbb{R}^n$ that follows \cref{eq:param-prop}, the composed function $\psi'\circ\psi^{-1}\colon\Xi\to\Xi'$ will be a $\mathit{C}^\infty$ diffeomorphism. By considering all the possible parametrizations of this kind we may treat $\mathcal{S}$ as a $\mathit{C}^\infty$-differentiable manifold, where statistical models are the charts and the different parametrizations are the coordinate systems; we call manifolds like these \emph{statistical manifolds}.

From our definitions, it is clear that the maximal dimension of a model is $n=N-1$ and that every statistical manifold is a submanifold of
\begin{equation}
    \mathcal{P}\coloneqq\{p\in\mathbb{R}^\mathcal{X}\mid p(x)>0 \quad \forall x\in\mathcal{X} \quad \text{and} \quad \sum_{x\in\mathcal{X}}p(x)=1\}
\end{equation}
that we call the \emph{manifold of probability distributions}. Notice that $\mathcal{P}$ is the interior of the space of probability distributions, this is because from our definitions follows that every $(N-1)$-dimensional statistical manifold must be an open subset of $\mathcal{A}_1$.
\todo{Esempi}

\section{The tangent space and its representations}
%Define m-representation of Tp through the displacement vector space of A1: A0
%Explain the natural isomorphism and state sum dip = 0 and dip l.i.
%Define the e-representation of Tp through the orthogonal space of the points
%Define the isomorphism justifying it as possible and saying it will be usefull that way later
We will now study tangent vectors of statistical manifolds looking for useful statistical interpretations of them. To do this we will use the fact that, as explained in \cref{ch:space-prob-dist}, $\mathcal{P}$ can be embedded in the space of random variables $\mathbb{R}^\mathcal{X}$. Then we can try to also embed the tangent spaces in $\mathbb{R}^\mathcal{X}$ in some meaningful way, thus linking tangent vectors and random variables.
